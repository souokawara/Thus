\documentclass[a4paper, twocolumn,10pt]{article}
\usepackage[p,osf]{scholax}
\usepackage{amsmath,amsthm}
\usepackage[scaled=1.075,ncf,vvarbb]{newtxmath}
\usepackage{graphicx}
\usepackage{mathtools}
\usepackage{tikz}
\usetikzlibrary {graphs,graphdrawing}
\usegdlibrary {trees}
\usetikzlibrary{calc, hobby}
\usepackage{float}
\author{So Okawara}
\date{\today}
\title{Image-rank Model}
\begin{document}
\maketitle
\section{Abstract}
\emph{Affection} is a directed graph on the network of images, where intention is path-dependency. The original idea is Google's \emph{Page-rank}. By the incidence matrix of nominal cues, the path of image association would be defined, on which nominal reasoning could base. Some image could have more gravity to be associated than another image, that could lay the foundations to speculate the \emph{intention} of the language.\\\\
The learning model for AI, which could know \emph{why} not only \emph{how}, by humanly association and \emph{recall}. Basically based on the methodology of Deep Learning.\\\\
There needs to be the demo window on which the internal images of AI streams as video. The learning would proceed on the multi-channel streaming videos. For the benchmarks, as a stand-alone it \emph{sleeps} to dream in non-prompt association, while \emph{free association} tests the by-prompt association. The window had better to associate audio speech with the inner streaming, as human does.\\\\
The learning on streaming should be based on Freud's pleasure principle, to formulate \emph{subjective values} on which Image-rank would also be calculated; thus, Image-rank will be not universal value system, rather client dependent, as each human also is to be an \emph{indivisual} in society.
\section{How to Associate Images on the Streaming Learning}
\section{Window of Mind}
\section{REM Sleep Test}
\section{Free Association Test}
\section{Audio Linkage to the Image Association}
\section{Speak more Freely by the Base of Image}
\end{document}